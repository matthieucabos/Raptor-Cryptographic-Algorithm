\documentclass{article}
\usepackage{amsmath}
\begin{document}
\title{Raptor Cryptographic Algorithm}
\newpage
\part{Crypting Protocol}

\begin{flushleft}
We will crypt a simple message containing the word 'salut'.

In a first step we have to compute the weight list of the differents caracters (meaning an approximation of the ASCII code used in the computer code algorithm).
\end{flushleft}

\section*{Weigth List}
\begin{flushleft}
Giving 0 to 'a' to 26 to 'z', we have : \textit{18.0.11.20.19} as the weigth list of the string
\end{flushleft}

\section*{Cumulated weigth list}

\begin{flushleft}
Once done, we have to compute the cumulated weigth list.
I mean, the list application can be considered as a suit defined by : 
\end{flushleft}

\begin{center}
$u_{n}$ a suit from N to N with the length $n$ $ \in $ N \quad $ \vert $ \quad $u_{i}$=$u_{i-2}$+$u_{i-1}$
\end{center}

In our case, the computed list is \textit{18.18.29.49.68}
We call it $v_{i}$
\section*{Key Computing}

\begin{flushleft}
At this moment we have to compute the public key $k_{i}$ of the algorithm defined via modulo since the formula :
\end{flushleft}

\begin{equation}
\centering
\left\{\begin{array}{@{}l@{}}
k_{i}=[u_{i}.u_{n-i} \quad mod \quad 26 ] + 10  \hspace*{1cm} if \quad \exists u_{i} , u_{n-i}  \\
\left. u_{j}  \; \hspace*{5cm} if \quad !\exists u_{j}, j = n/2 + 1
\right.\end{array}\,.
\right.\end{equation}

With our example, it gives :

\begin{equation}
\centering
\left\{\begin{array}{@{}l@{}}
k_{0}=[18.19 \quad  mod \quad  26] + 10 = 14 \\
k_{1}=[20.0 \quad  mod \quad  26] + 10 = 10 \\
k_{2}=[11 \quad  mod \quad  26] + 10 = 11 \\
\end{array}\right.\,.
\end{equation}

We build the full Length key $\xi$  \ using the formula :

\begin{equation}
\centering
\left\{\begin{array}{@{}l@{}}
\xi_{i} = k_{i} \hspace*{1cm} if \quad i <= n/2 + 1 \\ 
\xi_{i} = k_{n-i} \hspace*{1cm} if \quad i > n/2 + 1 \\ 
\end{array}\right.\,.
\end{equation}

\newpage

\section*{Crypting Process}
\begin{flushleft}
The crypting process is ruled by a pseudo-convolution with the given symbol $*$ meaning a point by point multiplication. 
This newer suit is ruled by $v_{i}$ and $u_{i}$
We call it $w_{i}$ defined by : $v_{i}$ $*$ $u_{i}$

In our example, it gives :
\end{flushleft}

\begin{equation}
\centering
\left\{\begin{array}{@{}l@{}}
w_{0}=v_{0}.u_{0}=324 \\
w_{1}=v_{1}.u_{1}=0 \\
w_{2}=v_{2}.u_{2}=319 \\
w_{3}=v_{3}.u_{3}=980 \\
w_{4}=v_{5}.u_{4}=1292 \\
\end{array}\right.\,.
\end{equation}

\begin{flushleft}
We obtain the suit $w$=\textit{324.0.319.980.1292}
\end{flushleft}

\section*{Encryption}
\begin{flushleft}
At the end we use the Encryption into differents numeric bases to hide the crypting process.

The Base indexes are defined by the key $ \xi $ \\
The list to encrypt is defined by $w$ \\
The Encryption process will be caled $ \Xi $ \\
Defined by : 
\begin{equation}
\Xi_{i}=(w_{i})_{\xi_{i}}
\end{equation}
\end{flushleft}

\begin{equation}
\centering
\left\{\begin{array}{@{}l@{}}
\Xi_{0} = (w_{0})_{\xi_{0}}=(324)_{14}=192 \\
\Xi_{1} = (w_{1})_{\xi_{1}}=(0)_{10}=0 \\
\Xi_{2} = (w_{2})_{\xi_{2}}=(319)_{11}=270 \\
\Xi_{3} = (w_{3})_{\xi_{3}}=(980)_{10}=980 \\
\Xi_{4} = (w_{4})_{\xi_{4}}=(1292)_{14}=684 \\
\end{array}\right.\,.
\end{equation}

\begin{flushleft}
The Encrypted suit is $ \Xi $ \ = \textit{192.0.270.980.684}

Its associate key is $ \xi $  \ = \textit{14.10.11.10.14}
\end{flushleft}

\newpage

\part{Decrypting Protocol}

\section*{Initialisation}


\begin{flushleft}
In this demonstration, we will use a Encrypted list using the Raptor cryptographic algorithm.
The terms list is given by :
\end{flushleft}

\begin{center}
!018kh"05a3c\#8064\$12vj\%2gai\&0605a(67500)0ba30*277a4+25376,2a5db-5813\.36u7!146367"27706\#1j68c
\end{center}

\begin{flushleft}
The associated key is given as a public key : 
\end{flushleft}

\begin{center}
2116103428141013
\end{center}

\begin{flushleft}
We consider in a first time differents type of caracters set used in the crypting and Encrypting processes.
\end{flushleft}

\begin{center}
\S = [!,",\#,\$,\%,\&,(,),*,+,-,\.]
\end{center}

Using this informations, we could get a first Terms list to treat called $ \Xi $.

\begin{center}
\textit{018kh.05a3c.8064.12vj.2gai.0605a.67500.0ba30.277a4.25376.2a5db.5813.36u7.146367.27706.1j68c}
\end{center}


\begin{flushleft}
	A list with length 16 is highlighting
We will use the Set X  = [a-z] $ \cup $  \ [0-9]

With $ \chi $\ the length of the Terms list.

Here $ \chi $ \ = 16, we could observ than length of key $ \rho  \quad  \vert \quad  \rho  = \chi . $

$ \Xi_{i} $ will represent the respectives terms of the list.

We start the decrypting process by exctracting the key's Bases index from the $c_{n}$ number suit contained in key.
with $ c_{i} $, $ \forall $ i $ \in $ [0,$ \rho $ \ ], $ c_{i} $ $\leq $ 9

\end{flushleft}

\begin{center}
We obtain : $\xi $ = \textit{21.16.10.34.28.14.10.13}
\end{center}
\newpage
\section*{Successive Base Transpositions - Step 1}

\begin{flushleft}
	Highlighted $\xi_{j}$ , Bases index are consistent with the Terms of the suit $\Xi$

Thereby, with the Correspondance between $ \xi_{0}$ and $\Xi_{0}$ , we obtain the following chained system resolution.
\end{flushleft}

\subsection{$\Xi_{0}$ = 018kh, $\xi_{0}$ = 21}
By drawing up the 21 Base Table, we find :
\begin{equation}
\centering
\left\{\begin{array}{@{}l@{}}
0=0 \\
1=1 \\
8=8 \\
k=20 \\
h=17 \\
\end{array}\right.\,.
\end{equation}

\begin{flushleft}
	Or by performing a Base transposition since the 21 Base Table, we obtain :
\end{flushleft}
\begin{equation}
(018kh)_{21}=(0.21^{4}+1.21^{3}+8.21^{2}+20.21+17)_{10}=13226
\end{equation}

\subsection{$\Xi_{1}$ = 05a3c, $\xi_{1}$ = 16}
By drawing up the 16 Base Table, we find :
\begin{equation}
\centering
\left\{\begin{array}{@{}l@{}}
0=0 \\
5=5 \\
a=10 \\
3=3 \\
c=12 \\
\end{array}\right.\,.
\end{equation}

\begin{flushleft}
	Or by performing a Base transposition since the 16 Base Table, we obtain :
\end{flushleft}
\begin{equation}
(05a3c)_{16}=(5.16^{3}+10.16^{2}+3.16+12)_{10}=23100
\end{equation}

\subsection{$\Xi_{2}$ = 8064, $\xi_{2}$ = 10}
The specified base index $\xi_{2}$ = 10, so any conversion is superfluous.


\subsection{$\Xi_{3}$ = 12vj, $\xi_{3}$ = 34}
By drawing up the 34 Base Table, we find :
\begin{equation}
\centering
\left\{\begin{array}{@{}l@{}}
1=1 \\
2=2 \\
v=31 \\
j=19 \\
\end{array}\right.\,.
\end{equation}

\begin{flushleft}
	Or by performing a Base transposition since the 34 Base Table, we obtain :
\end{flushleft}
\begin{equation}
(12vj)_{34}=(1.34^{3}+2.34^{2}+31.34+19)_{10}=42689
\end{equation}

\subsection{$\Xi_{4}$ = 2gai, $\xi_{4}$ = 28}
By drawing up the 28 Base Table, we find :
\begin{equation}
\centering
\left\{\begin{array}{@{}l@{}}
2=2 \\
g=16 \\
a=10 \\
i=18 \\
\end{array}\right.\,.
\end{equation}

\begin{flushleft}
	Or by performing a Base transposition since the 28 Base Table, we obtain :
\end{flushleft}
\begin{equation}
(2gai)_{28}=(2.28^{3}+16.28^{2}+10.28+18)_{10}=56746
\end{equation}

\subsection{$\Xi_{5}$ = 0605a, $\xi_{5}$ = 14}
By drawing up the 14 Base Table, we find :
\begin{equation}
\centering
\left\{\begin{array}{@{}l@{}}
0=0 \\
6=6 \\
5=5 \\
a=10 \\
\end{array}\right.\,.
\end{equation}

\begin{flushleft}
	Or by performing a Base transposition since the 14 Base Table, we obtain :
\end{flushleft}
\begin{equation}
(0605a)_{14}=(6.14^{3}+5.14+10)_{10}=16544
\end{equation}

\subsection{$\Xi_{6}$ = 67500, $\xi_{6}$ = 10}
The specified base index $\xi_{6}$ = 10, so any conversion is superfluous.


\subsection{$\Xi_{7}$ = 0ba30, $\xi_{7}$ = 13}
By drawing up the 13 Base Table, we find :
\begin{equation}
\centering
\left\{\begin{array}{@{}l@{}}
b=11 \\
a=10 \\
3=3 \\
0=0 \\
\end{array}\right.\,.
\end{equation}

\begin{flushleft}
	Or by performing a Base transposition since the 13 Base Table, we obtain :
\end{flushleft}
\begin{equation}
(0ba30)_{13}=(11.13^{3}+10.13^{2}+3.13+13)_{10}=25886
\end{equation}

The Base transposition done, we could reverse the key to obtain the rest of the list.
\section*{Key build}
\begin{flushleft}
We can use the following definition :

$\rho$ is the length of the key $\xi$ since Initialisation Section.

We go to compare the $\rho$ length of $\xi$ with $\chi$ the length of $\Xi$.We have $\chi$=2.$\rho$

We will use the following terms : 

\begin{itemize}
	\item $\tilde{\xi}$ : the mirror of $\xi$
	
	\item $\tilde{\xi}_{/n}$ : the mirror of $\xi$ bereft of $\xi_{n}$
	
	\item $\mathring{\xi}$ : the rebuilded key	
	
	\item $\frown$ : the concatenation operator
\end{itemize}
\end{flushleft}

\begin{flushleft}
To rebuild the missing half key, we go to reverse $\xi$ with the following syntax 	
\end{flushleft}

\begin{equation}
\centering
\left\{\begin{array}{@{}l@{}}
\mathring{\xi} = \xi^\frown  \tilde{\xi} \hspace*{4cm} if \quad \chi \quad mod \quad 2 = 0\\
\mathring{\xi} = \xi^\frown  \tilde{\xi}_{/n} \hspace*{3.7cm} if \quad \chi \quad mod \quad 2 = 1
\end{array}\right.\,.
\end{equation}

\section*{Successive Base Transpositions - Step 2}

Once the full key rebuilded from $\xi$, we could transpose again the rest of the list as step 1.

\subsection{$\Xi_{8}$ = 277a4, $\xi_{8}$ = 13}
By drawing up the 13 Base Table, we find :
\begin{equation}
\centering
\left\{\begin{array}{@{}l@{}}
2=2 \\
4=4 \\
7=7 \\
a=10 \\
\end{array}\right.\,.
\end{equation}

\begin{flushleft}
	Or by performing a Base transposition since the 13 Base Table, we obtain :
\end{flushleft}
\begin{equation}
(277a4)_{13}=(2.13{4}+7.13^{3}+7.13^{2}+10.13+4)_{10}=73818
\end{equation}

\subsection{$\Xi_{9}$ = 25376, $\xi_{9}$ = 10}
The specified base index $\xi_{9}$ = 10, so any conversion is superfluous.

\subsection{$\Xi_{10}$ = 2a5db, $\xi_{10}$ = 14}
By drawing up the 14 Base Table, we find :
\begin{equation}
\centering
\left\{\begin{array}{@{}l@{}}
2=2 \\
5=5 \\
a=10 \\
b=11 \\
d=13 \\
\end{array}\right.\,.
\end{equation}

\begin{flushleft}
	Or by performing a Base transposition since the 14 Base Table, we obtain :
\end{flushleft}
\begin{equation}
(2a5db)_{14}=(2.14{4}+10.14^{3}+5.14^{2}+13.14+11)_{10}=105 445
\end{equation}

\subsection{$\Xi_{11}$ = 5813, $\xi_{11}$ = 28}
By drawing up the 28 Base Table, we find :
\begin{equation}
\centering
\left\{\begin{array}{@{}l@{}}
1=1 \\
3=3 \\
5=5 \\
8=8 \\

\end{array}\right.\,.
\end{equation}

\begin{flushleft}
	Or by performing a Base transposition since the 28 Base Table, we obtain :
\end{flushleft}
\begin{equation}
(5813)_{28}=(5.28^{3}+8.28^{2}+1.28+3)_{10}=116 063
\end{equation}

\subsection{$\Xi_{12}$ = 36u7, $\xi_{12}$ = 34}
By drawing up the 34 Base Table, we find :
\begin{equation}
\centering
\left\{\begin{array}{@{}l@{}}
3=3 \\
6=6 \\
7=7 \\
u=30 \\
\end{array}\right.\,.
\end{equation}

\begin{flushleft}
	Or by performing a Base transposition since the 34 Base Table, we obtain :
\end{flushleft}
\begin{equation}
(36u7)_{34}=(3.34^{3}+6.34^{2}+30.34+7)_{10}=125 875
\end{equation}

\subsection{$\Xi_{13}$ = 146367, $\xi_{13}$ = 10}
The specified base index $\xi_{13}$ = 10, so any conversion is superfluous.

\subsection{$\Xi_{14}$ = 27706, $\xi_{14}$ = 16}

\begin{flushleft}
	Or by performing a Base transposition since the 16 Base Table, we obtain :
\end{flushleft}
\begin{equation}
(27706)_{16}=(2.16{4}+7.16^{3}+7.16^{2}+6)_{10}=161542
\end{equation}

\subsection{$\Xi_{15}$ = 1j68c, $\xi_{15}$ = 21}
By drawing up the 21 Base Table, we find :
\begin{equation}
\centering
\left\{\begin{array}{@{}l@{}}
1=1 \\
6=6 \\
8=8 \\
c=12 \\
j=19 \\
\end{array}\right.\,.
\end{equation}

\begin{flushleft}
	Or by performing a Base transposition since the 21 Base Table, we obtain :
\end{flushleft}
\begin{equation}
(1j68c)_{21}=(1.21{4}+19.21^{3}+6.21^{2}+8.21+12)_{10}=373266
\end{equation}

\begin{center}
	We finnaly obtain the following numeric suit :
	
\textit{13226.23100.42689.56746.16544.67500.25886.73818.25376.105445.116063.125875.161542.373266}
\end{center}

\section*{Chain Polynom Resolution}

To continue the decrypting process, we know the suit increasing by recurrence.
We can resolve the polynom using logic, we call it $Ch$.

\begin{center}
	$Ch_{n}$ = $y^{2}$+($y'^{2}$+($y''^{2}$+...+$y^{(n)2}$)).y + c =0
\end{center}
The recursive injection of a polynome is resolvable uniquely using positive real roots.

With this definition, we will not keep cases with $\triangle \leq 0 $ 

In the last part of the demonstration, we will use the Chain Polynoms resolution algorithm defined by :
\begin{itemize}
	\item Solve $ y^{2}+b.y-\Xi_{i}=0$
	\item $x=(root > 0) - b$
	\item $b = root$
	\item Add x to the solved list R.
\end{itemize}
\newpage
We gonna initialize the procedure with :
\begin{itemize}
	\item \begin{flushleft}
		$ y^{2}=\Xi_{0} \iff y=\sqrt{13226}=115$
		
	$R_{0}=115$
	\end{flushleft}
	\item \begin{flushleft}
	$ y^{2}-115.y-23100=0$
	
	$x=220-115=105$
	
	$R_{1}=105$
\end{flushleft}
	\item \begin{flushleft}
	$ y^{2}-220.y-8064=0$
	
	$R_{2}=252-220=32$
\end{flushleft}
\item \begin{flushleft}
$ y^{2}-252.y-42688=0$

$R_{3}=368-252=116$
\end{flushleft}
\item \begin{flushleft}
	$ y^{2}-368.y-56745=0$
	
	$R_{4}=485-368=117$
\end{flushleft}
\item \begin{flushleft}
	$ y^{2}-485.y-16544=0$
	
	$R_{5}=517-485=32$
\end{flushleft}
\item \begin{flushleft}
	$ y^{2}-517.y-67500=0$
	
	$R_{6}=625-517=108$
\end{flushleft}
\item \begin{flushleft}
	$ y^{2}-625.y-25896=0$
	
	$R_{7}=664-625=39$
\end{flushleft}
\item \begin{flushleft}
	$ y^{2}-664.y-73817=0$
	
	$R_{8}=761-664=97$
\end{flushleft}
\item \begin{flushleft}
	$ y^{2}-761.y-25376=0$
	
	$R_{9}=793-761=32$
\end{flushleft}
\item \begin{flushleft}
	$ y^{2}-793.y-105444=0$
	
	$R_{10}=909-793=116$
\end{flushleft}
\item \begin{flushleft}
	$ y^{2}-909.y-116622=0$
	
	$R_{11}=1023-909=114$
\end{flushleft}
\item \begin{flushleft}
	$ y^{2}-1023.y-125874=0$
	
	$R_{12}=1134-1023=111$
\end{flushleft}
\item \begin{flushleft}
	$ y^{2}-1134.y-146367=0$
	
	$R_{13}=1251-1134=117$
\end{flushleft}
\item \begin{flushleft}
	$ y^{2}-1251.y-161542=0$
	
	$R_{14}=1369-1251=118$
\end{flushleft}
\item \begin{flushleft}
	$ y^{2}-1369.y-373266=0$
	
	$R_{15}=1602-1369=118$
\end{flushleft}
\end{itemize}

\section*{Conclusion}

\begin{flushleft}
	we can conclude using a simple ASCII table and get letters from the obtained numeric suit.

R=\{115,105,32,116,117,32,108,39,97,92,116,114,11,117,118,233\}

$ASCII_{R}$=\{s,i, ,t,u, ,l,',a, ,t,r,o,u,v,é \}
\end{flushleft}

\begin{center}
	We can get the final decrypted string : "si tu l'a trouvé"
\end{center}



\end{document}